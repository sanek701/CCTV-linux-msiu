\chapter*{Введение}
\addcontentsline{toc}{chapter}{Введение}

Система видеонаблюдения~--- это программно-аппаратный комплекс
(видеокамеры, объективы, мониторы, регистраторы и др. оборудование),
предназначенный для организации видеоконтроля как на локальных, так и на
территориально-распределенных объектах.

В МГИУ используются цифровые камеры, установленные на этажах и в помещениях зданий университета.
Цифровые камеры позволяют передавать видеосигнал хорошего качества по уже существующей
сетевой инфраструктуре.

В настоящее время запущено порядка сотни камер. В связи с этим возникает необходимость
организовать запись видеосигнала с этих камер с возможностью архивации и просмотра ключевых событий.
Для этих целей создается специализированный сервер видеонаблюдения.

Под ключевыми событиями понимаются случаи обнаружения движения. Так как сами камеры не производят
анализа изображения, то эту функцию должен взять на себя сервер видеонаблюдения и программно
производить данный анализ. Информация о всех события должна быть сохранена и при необходимости
найдена и предоставлена пользователям системы.

Сервер видеонаблюдения должен располагать информацией о всех произведенных видеозаписях,
позволять быстро находить запись интересующего пользователя момента времени. В случае заполнения
дискового пространства производить автоматическую очистку архивных записей.

Доступ к сигналу с камер и к архивным видеозаписям должен осуществляться с помощью графического
интерфейса пользователя. Доступ напрямую к камерам и файлам видеозаписей должен быть исключен.
Настройка сервера и добавление камеры также должны производиться с помощью графического интерфейса.

В качестве серверной платформы следует использовать операционную систему GNU/Linux.
Графический интерфейс пользователя должен быть платформонезависимым.

Проанализировав существующие системы, было принято решение о необходимости написания
собственного сервера видеонаблюдения. Был выбран набор технологий и сторонних библиотек и программ,
используемых для реализации системы. Сервер реализован на языке Си с графическим интерфейсом,
построенным с использованием WEB технологий и фрейморка Ruby on Rails.
