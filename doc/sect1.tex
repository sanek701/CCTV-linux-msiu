\chapter{Литературный обзор}
\section{Обзор существующих решений}

Существующие решения можно разделить на несколько групп:
\smallskip
\begin{itemize}
	\item
	Видеорегистраторы~--- специализированные устройства,
	производящие запись и обработку видео с цифровых и/или аналоговых камер.
	\item
	Карты видеонаблюдения(платы видеозахвата)~--- устройства, производящие оцифровку
	сигнала с нескольких аналоговых камер. Как правило, способные самостоятельно сжимать
	видео в определенные форматы. Часто поставляются с примитивным ПО, позволяющим
	просматривать сигнал с камер.
	\item
	Программы, выполняющие обработку цифрового видео и работающие на оборудовании общего
	назначения.
\end{itemize}

\medskip

Для OC Linux существует всего 2 специализированные программы с открытым исходным кодом.

\subsection{Motion}

Аскетичное ПО в виде ядра с пристраиваемыми модулями и дополнениями.
Предназначена для работы в режиме <<демона>> (фонового процесса).

\medskip

Основные возможности:
\smallskip
\begin{itemize}
	\item
	работа с аналоговыми и IP-видеокамерами, работа с видео-серверами и камерами,
	транслируемыми из других серверов motion
	\item
	наличие встроенного веб-сервера, что позволяет обратиться к правильно
	настроенной видеокамере из любого веб-браузера
	\item
	запись при обнаружении движения или постоянно, в зависимости от настроек
	\item
	сохранение информации в виде отдельных картинок (MJPEG), MPEG-1, MPEG-4 и
	фактически в любом формате, который поддерживает ffmpeg
	\item
	возможность создания гибридных, аналогово-цифровых систем видеонаблюдения
	(очень удобно, если на предприятии еще есть старые, аналоговые видеокамеры
	и уже появляются цифровые)
	\item
	возможность исполнения программ при событии
	(например, отсылка письма с фотографией на указанный адрес при обнаружении движения)
	\item
	поддержка поворотных видеокамер
	\item
	создание маски неактивности (необходимость <<блокировать>> раскачивающиеся
	деревья или проезжающие машины по автостраде)
\end{itemize}

\medskip

Ограничения:
\smallskip
\begin{itemize}
	\item
	поддерживает только видеокамеры, транслирующие изображение в формате mjpeg (череда картинок)
	\item не поддерживает протокол RTSP
	\item не поддерживает запись с использованием современного видео кодека h264
	\item отсутствует возможность записи звука
	\item высокая сложность установки и настройки, необходимость уверенных знаний Linux
	\item отсутствует возможность отслеживать <<переполнение>> жесткого диска
	\item в настоящее время не развивается, крайний релиз от 1 июня 2010 года
	\item имеет большой список неисправленных проблем
\end{itemize}

\medskip

В МГИУ используются видеокамеры, транслирующие изображение в формате h264 по протоколу RTSP.
Исходный код запутан и плохо документирован.
Используемый подход к хранению видео в виде большого количества JPEG картинок не
отвечает требованиям, предъявляемыми к системе видеонаблюдения МГИУ.
Переработка данной программы крайне сложна и нецелесообразна.

\subsection{Zoneminder}

Написан на нескольких языках (perl, C, php).
Предоставляются удобные меню, интуитивно понятный web-интерфейс пользователя
(имеется поддержка русского языка), поддержка неограниченного числа аналоговых
и цифровых видеокамер. Для пользователей, незнакомых с Linux, предоставляется live-cd
дистрибутив с довольно простым и понятным интерфейсом для установки (на английском).

\medskip

Основные возможности:
\smallskip
\begin{itemize}
	\item
	работа с протоколом RTSP, h264, MJPEG, JPEG для IP-видеокамер
	\item
	работа с аналоговыми и цифровыми видеокамерами, работа с каналами,
	транслируемыми другими zoneminder или motion-системами
	\item
	наличие встроенного веб-сервера с возможностью трансляции изображения
	в режиме несколько камер на один экран
	\item
	запись при обнаружении движения, по расписанию
	\item
	возможность отсылки тревожной информации по электронной почте
	\item
	поддержка поворотных видеокамер
	\item
	множественный вход в систему под паролем с разделением списка просматриваемых камер
	и прав на управление системой по паролю
	\item
	отслеживание <<переполнения>> жесткого диска средствами самой системы
	\item
	поддержка ffmpeg и работа со всеми форматами, доступными данному набору библиотек
	\item
	создание маски неактивности
\end{itemize}

\medskip

Ограничения:
\smallskip
\begin{itemize}
	\item сложность установки и настройки системы, необходимы уверенные знания Linux
	\item отсутствует возможность записи звука
	\item видеоданные сохраняются в формате mjpeg
\end{itemize}

\medskip

Реализация модуля RTSP не совместима с камерами, которые используются в МГИУ. При тестировании
программа вела себя крайне нестабильно~--- часто аварийно завершала свою работу.
Архитектура системы сильно затрудняет доработку под условия использования в МГИУ.

\medskip

В связи с тем, что ни одна из систем не удовлетворила требованиям, которые предъявляются к системе
видеонаблюдения МГИУ, а также доработка данных систем является слишком сложной, было принято
решение разработать новую систему.


\section{Постановка задачи}

Спроектировать и реализовать программный комплекс для видеонаблюдения МГИУ. В распоряжении МГИУ
имеется порядка сотни IP видеокамер, передающие видео, закодированное кодеком h264 по протоколу RTSP.

\medskip

Требуется:
\smallskip
\begin{enumerate}
	\item Организовать круглосуточную запись сигнала с камер.
	\item Программными средствами обнаруживать движение.
	\item Сохранять информацию о событиях в базу данных.
	\item Предоставить возможность навигации по архиву видеозаписей без остановки процесса записи сигнала с камер.
	\item Создать интерфейс для администрирования системы.
	\item Создать интерфейс для организации рабочего места охранника.
	\item Предоставить возможность просматривать картинку с 1, 4, 9 или 16 камер одновременно.
	\item Предоставить возможность поиска событий.
	\item Для экономии места сжимать видео кодеком h264.
	\item Автоматически контролировать количество свободного дискового пространства и проводить очистку архивных записей при его заполнении.
\end{enumerate}

Управление сервером и просмотр видеозаписей должны быть возможны на машине с любой операционной
системой при помощи веб-браузера Firefox с установленным плагином VLC.


\section{Обзор используемых технологий}

\subsection{Протокол RTSP}
RTSP (Real Time Streaming Protocol, потоковый протокол реального времени)~--- это прикладной протокол,
в котором описаны команды для управления видеопотоком.
С помощью этих команд возможно управление камерой или медиа-сервером.
Пример команды: начать трансляцию видеопотока.

RTSP не выполняет сжатие, а также не определяет метод инкапсуляции мультимедийных данных и
транспортные протоколы. Передача потоковых данных сама по себе не является частью протокола RTSP.
Большинство серверов RTSP используют для этого стандартный транспортный протокол реального времени,
осуществляющий передачу аудио и видео данных(RTP).

По синтаксису и операциям протокол RTSP похож на HTTP. Однако между протоколами RTSP и HTTP есть ряд
существенных различий. Одно из основных заключается в том, что в первом и сервер, и клиент способны
генерировать запросы. Например, медиа-сервер может послать запрос для установки параметров
воспроизведения определенного видеопотока. Далее, протоколом RTSP предусматривается, что управление
состоянием или связью должен осуществлять сервер, тогда как HTTP таких возможностей не имеет.
Также в RTSP данные могут передаваться вне основной полосы другими протоколами,
например RTP, что невозможно в случае HTTP. RTSP-сообщения посылаются отдельно от мультимедийного
потока. Для них используется специальный порт с номером 554.

По протоколу RTSP передаются данные от камер медиа-серверу и от медиа-сервера клиентам.

\subsection{Протокол RTP}
Протокол RTP (англ. Real-time Transport Protocol) работает на прикладном уровне и используется
при передаче трафика реального времени.
Протокол RTP переносит в своем заголовке данные, необходимые для восстановления аудиоданных или
видеоизображения в приемном узле, а также данные о типе кодирования информации. В заголовке данного
протокола, в частности, передаются временная метка и номер пакета. Эти параметры позволяют при
минимальных задержках определить порядок и момент декодирования каждого пакета, а также
интерполировать потерянные пакеты.

RTP был разработан как протокол реального времени, из конца в конец (end-to-end), для передачи
потоковых данных. В протокол заложена возможность компенсации джиттера и обнаружения нарушения
последовательности пакетов данных~--- типичных событий при передаче через IP-сети. RTP поддерживает
передачу данных для нескольких адресатов через Multicast. RTP рассматривается как основной стандарт
для передачи голоса и видео в IP-сетях и совместно с кодеками.

Приложения, формирующие потоки реального времени, требуют своевременной доставки информации и для
достижения этой цели могут допустить некоторую потерю пакетов. Например, потеря пакета в
аудио-приложении может привести к доле секунды тишины, которая может быть незаметна при
использовании подходящих алгоритмов скрытия ошибок. Протокол TCP, хотя и стандартизирован для
передачи RTP, как правило не используется в RTP-приложениях, так как надежность передачи в TCP
формирует временные задержки. Вместо этого, большинство реализаций RTP базируется на UDP.

По запросу медиа-сервер создает RTP поток, который транслируется через RTSP сервер клиентам.

\subsection{Кодек H.264}
H.264~--- стандарт сжатия видео, предназначенный для достижения высокой степени сжатия видеопотока
при сохранении высокого качества.

Стандарт H.264 / AVC / MPEG-4 Part 10 содержит ряд новых возможностей, позволяющих значительно
повысить эффективность сжатия видео по сравнению с предыдущими (такими, как ASP) стандартами.

Потоки от камер кодируются кодеком H.264 и без перекодирования сохраняются видеосервером на
жестком диске. Видео, передаваемое клиентам также закодировано в H.264.

\medskip

Особенности:
\smallskip
\begin{itemize}
	\item Непроприетарный
	\item Хорошая степень сжатия
	\item Соответствие стандарту MPEG
\end{itemize}

\medskip

Недостатки:
\smallskip
\begin{itemize}
	\item требует значительных вычислительных ресурсов для кодирования и раскодирования
\end{itemize}

\subsection{Медиаконтейнер MP4}
Формат медиаконтейнера, являющийся частью стандарта MPEG-4.
Используется для упаковки цифровых видео- и аудиопотоков, субтитров, постеров и метаданных,
которые определены группой специалистов MPEG.
Как и большинство современных медиаконтейнеров, MPEG-4 Part 14 предусматривает возможность
осуществлять потоковое вещание через интернет, дополнительно к файлу передаются метаданные,
содержащие необходимую для вещания информацию.
Контейнер позволяет упаковывать несколько видео- и аудиопотоков, а также субтитров.

\subsection{JSON}
JSON~-- простой, основанный на использовании текста, способ хранить и передавать структурированные
данные. С помощью простого синтаксиса  можно легко хранить различные данные, начиная от одного числа
до строк, массивов и объектов в простом тексте. Также можно связывать между собой массивы и объекты,
создавая сложные структуры данных.

После создания строки JSON, ее легко отправить другому приложению или в другое место сети,
так как она представляет собой простой текст.

\medskip
JSON имеет следующие преимущества:
\smallskip
\begin{itemize}
	\item  компактен
	\item структуры данных легко читаются и составляются как человеком, так и компьютером
	\item легко преобразовать во внутренние структуры данных для большинства языков программирования
	(числа, строки, логические переменные, массивы и так далее)
	\item многие языки программирования имеют функции и библиотеки для чтения и создания структур JSON
\end{itemize}

JSON означает JavaScript Object Notation (представление объектов JavaScript). Как следует из названия,
он основан на способе определения объектов (очень похоже на создание ассоциативных массивов в других
языках) и массивов.
Наиболее частое распространенное использование JSON~--- пересылка данных от сервера к браузеру. Обычно
данные JSON доставляются с помощью AJAX, который позволяет обмениваться данными браузеру и серверу без
необходимости перезагружать страницу.

JSON используется в системе для обмена данными между HTTP сервером и веб-браузерами клиентов,
а также для отсылки команд медиа-серверу.

\section{Обзор используемых программ и библиотек}

\subsection{FFmpeg}
FFmpeg является проектом с открытым исходным кодом, в рамках которого создаются библиотеки и
программы для обработки мультимедийных данных.
Наиболее известными частями FFmpeg являются:
\smallskip
\begin{itemize}
	\item
	libavutil~--- содержит набор вспомогательных функций, которые включают в себя генераторы
	случайных чисел, структуры данных, математические процедуры, основные мультимедиа утилиты
	и многое другое
	\item
	libavcodec~--- содержит энкодеры и декодеры для аудио/видео кодеков
	\item
	libavformat~--- содержит мультиплексоры и демультиплексоры контейнеров мультимедиа
	\item
	libswscale~--- содержит хорошо оптимизированные функции для выполнения масштабирования
	изображений, преобразования цветовых пространств и форматов пикселов
	\item
	ffmpeg~--- консольная программа, предоставляющая доступ к возможностям библиотек
	\item
	ffserver~--- является HTTP и RTSP мультимедийным потоковым сервером, предназначенным
	для трансляций аудио и видео
\end{itemize}

\subsection{OpenCV}
OpenCV (Open Computer Vision)~--- библиотека компьютерного зрения с открытым исходным кодом,
предоставляющая набор типов данных и численных алгоритмов для обработки изображений алгоритмами
компьютерного зрения.

\medskip
Основные модули библиотеки:
\smallskip

\textbf{cxcore}
\smallskip
Ядро. Содержит базовые структуры данных и алгоритмы:
\smallskip
\begin{itemize}
	\item базовые операции над многомерными числовыми массивами
	\item матричная алгебра, математические функции, генераторы случайных чисел
	\item запись/восстановление структур данных в/из XML
	\item базовые функции двухмерной графики
\end{itemize}

\medskip
\textbf{CV}
\smallskip
Модуль обработки изображений и компьютерного зрения:
\smallskip
\begin{itemize}
	\item
	базовые операции над изображениями (фильтрация, геометрические преобразования,
	преобразование цветовых пространств и т. д.)
	\item
	анализ изображений (выбор отличительных признаков, морфология, поиск контуров, гистограммы)
	\item
	анализ движения, слежение за объектами
	\item
	обнаружение объектов, в частности лиц
	\item
	калибровка камер, элементы восстановления пространственной структуры
\end{itemize}

\medskip
\textbf{Highgui}
Модуль для ввода/вывода изображений и видео, создания пользовательского интерфейса:
\smallskip
\begin{itemize}
	\item захват видео с камер и из видео файлов, чтение/запись статических изображений
	\item функции для организации простого пользовательского интерфейса
\end{itemize}

\medskip
OpenCV используется для анализа кадров на наличие движения, а также для создания изображений.

\subsection{Gstreamer}
Gstreamer~--- мультимедийный фреймворк, написанный на языке программирования C и
использующий систему типов GObject. GStreamer является <<ядром>> мультимедийных приложений,
таких как видеоредакторы, потоковые серверы и медиаплееры. В изначальный дизайн заложена
кроссплатформенность.

Системой используется реализация RTSP сервера на базе Gstreamer для вещания видео клиентам сервера
видеонаблюдения.

\subsection{PostgreSQL}
PostgreSQL~--- это полноценная SQL СУБД с большим списком возможностей и огромным количеством
людей по всему миру, которые используют и разрабатывают эту СУБД.

PostgreSQL ориентировалась на использование в сложных приложениях. Именно поэтому упор всегда делался
на надежность, наличие развитой функциональности и соответствие стандартам. При этом, PostgreSQL
можно точно также использовать и в веб-приложениях, где данная СУБД показывает неизменно отличные
результаты, при лучшей масштабируемости и настраиваемости.

\subsection{libpq}
Си интерфейс к PostgreSQL. libpq~--- набор функций, позволяющих клиентским программам
отправлять запросы к серверам PostgreSQL и получать результаты их выполнения.

PostgreSQL используется для хранения настроек камер, информации и событиях и создаваемых видеофайлах.
Видеосервер использует библиотеку libpq для доступа к базе данных.

\subsection{libconfig}
libconfig~--- библиотека для обработки конфигурационных файлов.
Позволяет обрабатывать удобочитаемые и легко редактируемые конфигурационные файлы.
Имеет малый размер.

Система использует библиотеку libconfig для обработки конфигурационного файла и получения настроек,
необходимых для запуска приложения.

\subsection{Ruby on Rails}
Ruby on Rails~--- фреймворк для веб-разработки, написанный на языке программирования Ruby.
Он разработан, чтобы сделать программирование веб-приложений проще, так как использует ряд допущений
о том, что нужно каждому разработчику для создания нового проекта. Он позволяет вам писать меньше
кода в процессе программирования, в сравнении с другими языками и фреймворками. Основан на
шаблоне проектирования Модель-Представление-Контроллер (Model-View-Controller, MVC).

Фрейморк используется для создания графического интерфейса пользователя.

\subsection{VLC}
VLC (от VideoLAN Client)~--- это бесплатный, открытый, кросс-платформенный мультимедиа
проигрыватель и фреймворк, который воспроизводит большинство мультимедийных файлов,
а также DVD, Audio CD, VCD и различные протоколы потокового вещания.

Проигрыватель имеет плагин для популярных браузеров, позволяющий использовать VLC для воспроизведения
различного медиа на web страницах. Плагин поддерживает все форматы, что и основная программа.

Плагин VLC для браузера Firefox используется для отображения видеосигнала от видеосервера.
